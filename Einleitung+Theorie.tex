\section{Einleitung}


In der heutigen Zeit möchte man immer komplexere Metallteile am Stück formen, da dies, wegen weniger potentieller Bruchstellen, zu höherer Stabilität führt. Besonders ist man daran interessiert das jetzige Warmformgebungsverfahren bei hohlen Aluminiumprofilen auch für Stahlprofile nutzbar zu machen. \\
Dabei steht man vor der Herausforderung, das man bei den Temperaturen der Stahl Warmformgebung keine Ölmischung mehr nutzen kann, da sich diese entzünden würde. \\
Aus diesem Grund wählte man als Medium die Stoffklasse der granularen Medien aus. Diese bringen allerdings den Nachteil mit sich, das sie kompressiebel sind und die an einer Seite aufgebrachte Kraft sich nicht homogen und isotrop auf den Wänden verteilt. 
Man nimmt an, das die Art der Kraftverteilung durch drei Hauptparameter charakterisieren lässt. Zum einen ist das die Form der verwendeten Partikel, die Größe der Partikel und die Adhäsionskrafte der Partikel untereinander. \\
Diese drei Parameter kann man gesammelt betrachten, indem man die granularen Medien in einem Wirbelbett fluidisiert und misst ab wann die Medien in die flüssige Phase übergehen. Dadurch erreicht man eine Vergleichbarkeit der Medien untereinander und kann im Anschluss besser entscheiden welche Medien sich als Kraftüberträger eignen. \\
\hfil \\
Diese Arbeit behandelt die Neukonzeption eines bestehenden Wirbelbettes am Deutschen Zentrum für Luft- und Raumfahrt, Institut für Materialphysik im Weltraum. Dabei geht es vor allen Dingen darum mit welchen konstruktiven Lösungen die Mängel des bisherigen Wirbelbettes behoben wurden. Weiterhin werden Testmessungen gemacht, um die Funktion des Wirbelbettes nachzuweisen.































