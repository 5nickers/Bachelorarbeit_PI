\chapter{Bewertung und Ausblick}


\section{Bewertung}

In diesem Abschnitt soll der Erfolg der einzelnen Arbeitspakete kritisch hinterfragt werden, um Raum zur Optimierung aufzuzeigen. 

\subsection{Elektronik}

Rückblickend wurde der Arbeitsaufwand für die Elektronik unterschätzt, was dazu führte, dass sehr lange unklar war, welche Teile verwendet werden. Dadurch war es nicht möglich, in dem geplanten Zeitfenster ein finales Gehäuse für die Elektronik zu designen. Da bis zum Schluss keine finale Version der Elektronik vorlag, war es nicht möglich ein Gehäuse zu fertigen. In Zukunft wäre es besser, solche Probleme entweder früher zu erkennen oder das Design von vorne herein ans Ende zu stellen. 

\subsection{Gassystem}

Der gewählte Filter hat einen Nachteil. Trotz planem Aufkleben des Filters auf das Probenröhrchen, verformt sich dieser im Laufe der Nutzung plastisch. Das führt zu einem zusätzlichen Heben und Senken des granularen Mediums um bis zu \SI{2}{cm} (im Röhrchen mit \SI{30}{mm}  Durchmesser). Dieses Verhalten ist unerwünscht, da sich die Granulate deswegen nicht komprimieren und dann auf ihr Verhalten untersuchen lassen. Dieses Verhalten widerspricht allerdings nicht den ursprünglichen Anforderungen, die alle erfüllt wurden.

\subsection{Wirbelbett}

Die Konstruktion des Wirbelbettes erfüllt alle Anforderungen und verlief fast optimal. Lediglich gegen Ende musste, auf Grund der Umstellung von Kunststoff (ABS) auf Metall, die Konstruktion in einigen Punkten abgeändert werden. Das führte zu Zeitproblemen, da die Werkstatt zu dem Zeitpunkt ausgelastet war und nicht alle Röhrchenhalter rechtzeitig fertig wurden.
Im Vergleich mit der ursprünglichen Version wurde ein hochwertiges und quantitativ verwendbares Wirbelbett gebaut.

\subsection{Nutzung des 3D Druckers}

Durch die Anforderung, einen Ionisator im Aufbau unterzubringen, war es nicht möglich ein anderes Material außer Plastik zu verwenden. Zudem konnte eine andere Herangehensweise gewählt werden. Statt klassischen technischen Zeichnungen und FEM, wurde die Methode des evolutionären Druckens gewählt. Das sparte die aufwändigen Simulationen und erlaubte das Testen der designten Teile unter Realbedingungen. Der Nachteil dieser Methode ist neben den  Fehldrucken, die bei ca. \SI{10}{\%} lagen, der Ausschuss an Teilen. Dadurch, dass man dasselbe Teil in verschiedenen Evolutionsstufen druckt, ist der Materialverbrauch recht hoch. \\
Die Qualität der gedruckten Bauteile war durchweg hoch, besonders die Stabilität ist beachtlich. Allerdings sind zwei oder drei Drucke notwendig, bis man sein Modell am Rechner soweit angepasst hat, dass der Drucker so druckt, wie man es beabsichtigt. Zudem ist der Drucker anfällig in Bezug auf den Druckuntergrund. Während des Projekts ist das Kaptontape auf der Heizplatte kaputtgegangen. Ohne dieses Tape auf der Heizplatte, ist es nicht möglich Teile mit Durchmesser größer als \SI{40}{mm} mit flachem Boden zu drucken. Die Drucke bogen sich und waren somit ungeeignet, für dichtende Zwecke eingesetzt zu werden. Weiterhin stellte sich im Rahmen der Messungen heraus, dass der gedruckte Anschlusszylinder als auch der Röhrchenhalter nicht luftdicht ist. Die genauen Ursachen sind nicht untersucht worden, allerdings werden die nicht optimal miteinander verschmolzenen Außenwände dabei eine große Rollen spielen.
Ein Gruppenmitglied hat empfohlen das gedruckte Teil für knapp eine Minute in Aceton zu tauchen. Dabei wird die oberste Schicht des ABS aufgeschmolzen und versiegelt sich beim trocknen wieder. Das Ergebnis ist eine glasartige Oberfläche, die eine gute Luftdichtigkeit aufweisen könnte. 


\subsection{Zusammenfassung}

Abschließend kann gesagt werden, dass das Projekt sehr erfolgreich abgeschlossen wurde. Es wurden alle Anforderungen erfüllt und es ist nun möglich Granulate quantitativ zu vermessen.
Im Verlauf der Konstruktion wurden viele neue Erkenntnisse über das Verhalten verschiedener Tools und Materialien gewonnen. Dies sollte genutzt werden, um bei zukünftigen Projekten in diesem Bereich genauer planen und Risiken und Problemstellen besser einschätzen zu können.




\section{Ausblick}

Der Ausblick ist zweigeteilt. Im ersten Teil wird auf das konstruierte Wirbelbett eingegangen und im zweiten Teil über dessen Weiterentwicklung diskutiert. 

Wie bereits in der Bewertung erwähnt, gibt es noch Raum für Verbesserungen bei einzelnen Komponenten. Dies betrifft zum Beispiel den Filter, der entweder ersetzt werden sollte oder durch konstruktive Maßnahmen daran gehindert werden sollte, dass er sich biegen kann. In diesem Zusammenhang kann man sich auch nochmals mit der Integration eines Luftionisator beschäftigen, da dies ein interessantes Konzept ist- Unser Ionisator hatte allerdings entweder zu wenig Leistung hatte oder der Filter ließ zu wenig Ionen durch. \\
Weiterhin wäre es interessant zu untersuchen, ob und wie es möglich ist die 3D gedruckten Teile luftdicht zu machen. Eine Möglichkeit die erwogen wurde, ist das Tränken des gedruckten Teils in Epokzitharz. Bei einem Versuch mit Z-Bond 90 war die Behandlung nicht erfolgreich und der Luftduchfluss verschlechterte sich nocheinmal. Aceton hingegen schien vielversprechender zu sein. Hier wäre es interessant herauszufinden welche Materialien für eine luftdichte Versiegelung in Frage kommen und wie das gedruckte Teil damit behandelt werden muss. Der Vorteil wäre ein deutlich leichteres Wirbelbett.

Die Weiterentwicklung des Wirbelbetts bestünde darin, es auch mit Wasser betreiben zu können. Dies hat den Vorteil, dass durch den höheren Auftrieb des Granulats im Wasser und durch die höhere Viskosität des Wassers, deutlich weniger Strömungsmedium gebraucht wird. Im Gegensatz zu Luft mit bis zu \SI{3}{m^3/h}, würde es unter \SI{1}{l/h} Wasser benötigen, um die Teilchen zu fluidisieren. Zudem hat man bei der Verwendung von Wasser kein Problem mit Elektrostatik. \\
Der Nachteil von Wasser als Strömungsmedium besteht darin, das man eine Rückführung braucht und das Wasser im Kreislauf auch sauber gehalten werden muss. Zudem wäre eine neue Regelungstechnik für Wasser nötig, sowie ein neuer Drucksensor. Weiterhin führt die Verwendung von Wasser als Fluidisationsmedium zu einer geänderten Dynamik des Granulats und zu einer veränderten Wechselwirkung. Der Einfluss dessen auf die Messung ist bis jetzt unklar.  \\
Eine Umsetzung eines Wirbelbetts auf Wasserbasis wäre eine gute Folgebachelorarbeit.


