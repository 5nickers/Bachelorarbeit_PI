\chapter{Praxisteil}

\section{AP 2 - Elektronik}

\subsection{Anforderungen}

Bei der Elektronik gab es zwei verschiedene Bereiche: \\
In einem ersten Schritt sollte ein A-D Wandler gefunden werden, damit der vorhandene Drucksensor genauer ausgelesen werden kann. Zum Start es Projekts wurde der Drucksensor über das $\SI{10}{bit}$ breite analoge Interface des Arduino ausgelesen. Das ist für eine genaue Auswertung der Experimente zu wenig. Der Drucksensor hat einen Messbereich von (nachgucken) und gibt diesen auf $\SI{0}{} - \SI{10}{V}$ aus. Da der Arduino maximal $\SI{5}{V}$ als Eingangsspannung verarbeiten kann, wird die Spannung über einen Spannungsteiler auf $\SI{0} - \SI{5}{V}$ reduziert und dann eingelesen. Das entspricht einer Genauigkeit von $\SI{0,004}{V/bit}$, was einer Genauigkeit von XXX mbar/Volt entspricht. \\
\hfill \\
Als Resultat soll die gesamte Elektronik als \glqq Blackbox\grqq \ vorliegen, sodass man nach außen hin nur noch Anschlüsse hat, die klar gekennzeichnet sind. Dies soll in Form eines Gehäuses erledigt werden, das zudem auch noch rudimentären Schutz gegen Schmutz bietet.


\subsection{Umsetzung}

\subsubsection{Elektronik}

Der Arduino bietet einen sehr einfachen SPI Bus zum Anschluss von Modulen an. Darüber können die digitalen Outputdaten, der angesteckten Module ausgelesen werden, unabhängig davon wie hoch die Auflösung der Module ist. Auf Grund dieser Voraussetzungen wurde entschieden ein Modul für diesen Bus zu verwenden. \\
Rechnung für 16 bit einfügen!!!!!! \\
Es gibt verschiedene $\SI{16}{bit}$ A-D Wandler auf dem Markt, allerdings gibt es den hier ausgewählten (einfügen!!!!) bereits auf einer fertig montierten Platine, die direkt in den Arduino eingesteckt werden kann. Zudem gibt es zu der Platine eine recht gute Dokumentation. 


Schaltskizze!!!

\subsubsection{Gehäuse}

Da es insbesondere im Lichtstreuaufbau nur sehr begrenzten Stauraum gibt und sämtliche Bauteile, das Wirbelbett selbst ausgenommen, auf einer Platte verbaut werden sollen, wurde gegen ein kommerziell erhältliches Gehäuse entschieden. \\ 
Stattdessen wurde ein Gehäuse entworfen, in dem sowohl der Arduino als auch das Board mit der Schaltung passgenau Platz finden. Dabei wurde darauf geachtet, dass die Kontakte unterhalb des Arduinos und des Boards genug Platz haben, sodass es keinerlei unbeabsichtigte Reibung der Komponenten am Gehäuse gibt. \\
Für die Anschlüsse wurden möglichst kleine Löcher im Gehäuse gelassen, sodass ein größtmöglicher Schutz gegen Schmutz bei größtmöglichem Komfort gewährleistet wird. \\
Um die Komplexität so gering wie möglich zu halten, wurde sich gegen einen Schließmechanismus entschieden, da dieser entweder anfällig oder unnötig teuer werden würde. Da das Gehäuse lediglich zum entfernen der angeschlossenen Kabel geöffnet werden muss, war es am einfachsten, den Deckel mit Klebeband am Gehäuse fest zu machen.

Hier Bilder einfügen!!


\section{AP3 - Gassystem}

\subsection{Anforderungen (kommt das hier hin?)}

Die Aufgabe bestand darin, trotz der Integration des Luftbefeuchters, das Gassystem als Einheit so kompakt und stabil wie möglich zu gestalten. Es sollte zusammen mit der Elektronik zusammen in den Lichtstreuaufbau passen und dort die Rotationbewegung des Aufbaus nicht einschränken.


\subsection{Umsetzung}

\subsubsection{Flowcontroller}

Um die Anforderung der Erweiterung des Regelbereichs beim Gasstrom auf $\SI{0} - \SI{3000}{l/h}$ zu erfüllen, musste der bisherige Flowcontroller mit einem Regelbereich von $\SI{0} - \SI{120}{l/h}$ ersetzt werden. \\
Die Idee einer konstruktiven Lösung den Strom über mehrere Gasdurchflussbegrenzer und Ventile auf definierte Werte voreinzustellen und mit dem vorhandenen Flowcontroller von da ab zu regeln, wurde verworfen, weil es sich als zu ungenau und nicht deutlich billiger in Aufwand und Kosten darstellte. \\
Die Lösung bestand darin einen neuen Flowcontroller anzuschaffen, der den gesamten Regelbereich mit einer hinreichenden Genauigkeit abdeckt und gleichzeitig gut in das Gassystem integriert werden kann. Dazu wurden Angebote eingeholt, die im folgenden dargelegt sind:

\begin{tabular}{|c|c|c|}
	\hline  & Brooks & Bronkhorst \\ 
	\hline Regelbereich & $\SI{0} - \SI{3000}{l/h}$ & $\SI{0} - \SI{3000}{l/h}$ \\ 
	\hline Genauigkeit $20 - \SI{100}{\%}$ & $\pm \SI{0,9}{\%}$ & $\pm \SI{0,5}{\%}$ Istwert $+ \pm \SI{0,1}{\%}$ Endwert\\ 
	\hline Genauigkeit $0 - \SI{20}{\%}$ & $\pm \SI{0,18}{\%}$ & $\pm \SI{0,5}{\%}$ Istwert $+ \pm \SI{0,1}{\%}$ Endwert \\ 
	\hline Preis in \euro & 1187,33 & 1382,66 \\ 
	\hline 
\end{tabular} 

\vspace{0,5cm}

Es wurde sich für den Flowcontroller der Firma Brooks entschieden. Das geschah aus mehreren Gründen: \\
Im ursprünglichen Aufbau war auch ein Flowcontroler von Brooks verbaut, daher wussten wir wie dieser angesteuert wird und hatten bereits den benötigten Stecker. Weiterhin ist die Genauigkeit des Brookscontrollers im niedriegen Regelbereich besser als die des Bronkhorstcontrollers. Dies ist grade bei quantitativen Messungen mit kleinen Rohrdurchmessern und sehr feinen granularen Medien essentiell. \\
Zudem waren die Anschaffungskosten für den Brookscontroller um ca $\SI{200}{Euro}$ niedriger.\\
Außerdem hatte die Gruppe bei einem anderen Controller von Bronkhorst einige Probleme mit der Zuverlässigkeit gehabt, während der Brooks Controller keine aufwies.















